% Template for computer science thesis at the TU Munich
% Authors Benedikt Mas y Parareda, Johannes Becker, Gunnar Schroeder, Elmar Juergens

\documentclass[11pt,a4paper]{article}

%Enable DVI forward search
%\usepackage[active]{srcltx}

\usepackage{multicol}
\usepackage[utf8x]{inputenc}
\usepackage[T1]{fontenc}
\usepackage{amsmath,amssymb,amsfonts}
\usepackage{color}
\usepackage{graphicx}
\usepackage{rotating}
\usepackage{listings}
\usepackage{cite}
\usepackage{url}
\usepackage{latexsym}
\usepackage{makeidx}
\usepackage{color}
\usepackage{natbib}
\usepackage{array}
\usepackage{todonotes}

\lstset{
        basicstyle=\scriptsize\ttfamily, % Standardschrift
        numbers=left,               % Ort der Zeilennummern
        numberstyle=\tiny,          % Stil der Zeilennummern
        %stepnumber=2,               % Abstand zwischen den Zeilennummern
        numbersep=5pt,              % Abstand der Nummern zum Text
        tabsize=2,                  % Groesse von Tabs
        %extendedchars=true,         %
        breaklines=true,            % Zeilen werden Umgebrochen
        keywordstyle=\color{red}\ttfamily,
   		frame=b,         
%        keywordstyle=[1]\textbf,    % Stil der Keywords
 %       keywordstyle=[2]\textbf,    %
  %      keywordstyle=[3]\textbf,    %
   %     keywordstyle=[4]\textbf,   \sqrt{\sqrt{}} %
        stringstyle=\color{blue}\ttfamily, % Farbe der String
        showspaces=false,           % Leerzeichen anzeigen ?
        showtabs=false,             % Tabs anzeigen ?
        xleftmargin=17pt,
        framexleftmargin=17pt,
        framexrightmargin=5pt,
        framexbottommargin=4pt,
        %backgroundcolor=\color{lightgray},
        %showstringspaces=false,      % Leerzeichen in Strings anzeigen ?        
		language=Java
}
\lstloadlanguages{Java}
\usepackage{caption}
\DeclareCaptionFont{white}{\color{white}}
\DeclareCaptionFormat{listing}{\colorbox[cmyk]{0.88, 0.44, 0.0,0.0}{\parbox{\textwidth}{\hspace{15pt}#1#2#3}}  }
\captionsetup[lstlisting]{format=listing,labelfont=white,textfont=white, singlelinecheck=false, margin=0pt, font={bf,footnotesize}}

\definecolor{darkgreen}{cmyk}{0.7, 0, 1, 0.5}
\definecolor{darkblue}{rgb}{0.1, 0.1, 0.5}

\lstdefinelanguage{diff}
{
keywords={+, -, \ , @@, diff, index, new},
sensitive=false,
morecomment=[l][""]{\ },
morecomment=[l][\color{darkgreen}]{+},
morecomment=[l][\color{red}]{-},
morecomment=[l][\color{darkblue}]{@@},
morecomment=[l][\color{darkblue}]{diff},
morecomment=[l][\color{darkblue}]{index},
morecomment=[l][\color{darkblue}]{new},
morecomment=[l][\color{darkblue}]{similarity},
morecomment=[l][\color{darkblue}]{rename},
}

\makeindex


\usepackage{geometry,mflogo,xspace,texnames,path,booktabs,bm}
\usepackage[hyperindex,bookmarks,pdfborder=0,plainpages=false,pdfpagelabels]{hyperref}


%Settings applicable to the complete document

%Breaks URLs properly and draws them in a nice font
\urlstyle{sm}

\renewcommand{\ttdefault}{pcr} % Courier has a bold shape, default tt does not


%Decrease the default indentation of paragraphs
\parindent=0.3cm

%new or changed commands
\renewcommand\contentsname{Table of Content}
\newcommand{\chapref}[1]{Chapter~\ref{#1}}
\newcommand{\secref}[1]{Section~\ref{#1}}
\newcommand{\appref}[1]{Appendix~\ref{#1}}
\newcommand{\tabref}[1]{Table~\ref{#1}}
\newcommand{\figref}[2][]{Figure~\ref{#2}#1}
\newcommand{\listref}[2][]{Listing~\ref{#2}#1}

\newcommand{\UH}{\textit{Unknown Horizons}}
\newcommand{\OS}{open-source}
\newcommand{\BOW}{\textit{Battle for Wesnoth}}
\newcommand{\AD}{\textit{0 A.D.}}
\newcommand{\GLEST}{\textit{Mega Glest}}

\begin{document}
%\bibliographystyle{plainnat}

%Title Page
\pagestyle{empty}

\begin{titlepage}
\begin{center}

\begin{figure}[!htb]
	\begin{center}
		\includegraphics[scale=0.50]{pics/ntnu}
	\end{center}
\end{figure}
\begin{figure}[!htb]
	\begin{center}
		\includegraphics[scale=0.40]{pics/idi}
	\end{center}
\end{figure}
\begin{LARGE}
\vspace{1.2in}

\end{LARGE}
\begin{Huge}
Report: Poker AI Implementation in Java
\vspace{1.7in}


\end{Huge}
\begin{LARGE}
Henning Funke \& Thomas Kinnen \vspace{0.6in}
\end{LARGE}

\begin{Large}
IT3105 - Kunstig intelligens programmering
\vspace{0.8in}
\end{Large}



\end{center}

\end{titlepage}

%% Abstract
\chapter*{Abstract}
This is the abstract, done in the end.


%\input{acknowledgement.tex}


% Table of Contents
\setcounter{tocdepth}{1}                % Sets depth of table of contents. 0 is chapter, 1 is sections, 2 is subsections
\setcounter{secnumdepth}{2}             % Sets depth of numbering of toc contents
\tableofcontents

\pagestyle{headings}

% Chapters. Each one in its own file
% {{{=================== Introduction ======================


\section{Introduction}

\subsection{Motivation}

%}}}


\section{Opponent Modelling}
In this section we explain our opponent model and the contexts on which it is based.

\subsection{Model Details}
We have one model per player, which every other player can access. The model contains a mapping for \textit{context} ->
\textit{handstrength} pairs played by the player. The pairs are only recorded if the player is part of the showdown at
the end of a single game. If a context appears multiple times, the average of all recorded handstrengths is used. All
context-handstrength pairs are saved in a \textit{HashMap} for very fast access to the data.

\subsection{Context Details}
We use fairly specific contexts. Our contexts carry the following information to identify them:
\begin{itemize}
    \item The game-state (PreTurn, PreRiver, etc.)
    \item The action the player playen (raise, call, fold)
    \item The player this context is valid for
    \item The common cards' values on the table, the suite is ignored
    \item The current pot-odds, devided into four different bins
    \item The number of raises that have been played in this betting round
\end{itemize}
Having the game-state and the current action in the context seemed natural, as they cleary differentiate the different
situations inside a game and should thus be treated by different contexts. As we expect to have different behaviour per
player, we included the player into the context.

The common cards on the table are one of the main differences between every round, thus it is essential to have this
information present in each context. Because there are many possible combinations of suites for the same values of
cards, we ignore the suites for the common cards. This is a basic form of equivalence class seperation.

The pot-odds is discretized into four different bins: $pot-odds < 0.1$, $pot-odds < 0.2 $, $pot-odds < 0.3$
and $ pot-odds \geq 0.3$. As the pot-odds is a double value, using the exact value would lead to almost infinite numbers of contexts, which
is not very useful for the AI to base decisions upon. As the pot-odds provide a lot of information concerning the risk
the player has when choosing an action, we think that it can be benificial to use it to differentiate between different
contexts.

We also included the number of raises that have been made in a round into the context as we believe that the number of
raises is a good indication of the general game situation, many raises implies that many players believe to have many
cards. This can influence the players decision on which action to take and should thus be included in the context.

\pagebreak

\section{Simulation Results}
In this chapter we present the results of our simulations. For each phase we ran five simulations of 2000 hands each.

\subsection{Phase I}
The results in phase I are very random. This was to be expected as the betting behavior of the phase I players is mainly
random.

\begin{table}[!h]
\center
\caption{Phase I Results - 5 simulations of 2000 hands each}
\begin{tabular}{ c | c c c c c}
    \textbf{Simulation} & 1 & 2 & 3 & 4 & 5 \\
    \hline
    \textbf{Player} \\
    Phase I - raiser - 1 & 456 & -5212 & -17941 & 22541  & -1153\\
    Phase I - raiser - 2 & 7700 & -1936 & 28227 & 6591   & 1720\\
    Phase I - raiser - 3 & -20680 & -537 & 9323 & -18126 & -18186\\
    Phase I - simple - 1 & -6776 & 11900 & 2034 & -39223 & 9172\\
    Phase I - simple - 2 & 8223 & -633 & -6517 & 14217   & 4860\\
    Phase I - simple - 3 & 17031 & 2360 & -9185 & 19965  & 9534\\
\end{tabular}
\end{table}

\subsection{Phase II}
In phase II we can clearly see that the players using rollout simulation results are much better than the random
players. Adding handstrength calculations results in an even greater gain and makes the player always win against his
simpler opponents.

\begin{table}[!h]
\center
\caption{Phase II Results - 5 simulations of 2000 hands each}
\begin{tabular}{ c | c c c c c}
    \textbf{Simulation} & 1 & 2 & 3 & 4 & 5 \\
    \hline
    \textbf{Player} \\
    Phase I - raiser        & -19305 & -8125 & -15256 & -17774 & -8243\\
    Phase I - simple        & -6707 & -13404 & -11603 & -4350  & -15067\\
    Phase II - only rollout & 8627  & 6346  & 2651    & 3535   & 10016\\
    Phase II - rollout + HS & 21335 & 19145 & 28160   & 22528  & 17247\\
\end{tabular}
\end{table}

\subsection{Phase III}
In Phase III players using an opponent model were added. This did not lead to an improved performance compared to the
Phase II players.

\begin{table}[!h]
\center
\caption{Phase III Results - 5 simulations of 2000 hands each}
\begin{tabular}{ c | c c c c c}
    \textbf{Simulation} & 1 & 2 & 3 & 4 & 5 \\
    \hline
    \textbf{Player} \\
    Phase I - simple            & -41870 & -62525   & -46869 & -77551 & -44061\\
    Phase II - only rollout     & -4884 & 16297     & 2030   & 4480  & -3097\\
    Phase II - rollout + HS     & 42133 & 28763     & 40565  & 42673 & 36456\\
    Phase III - Modelling       & 3067  & 23382     & 24950  & 31286 & 32294\\
    Phase III - Modelling V2    & 6468  & -977      & -15717 & 4050  & -16663\\
\end{tabular}
\end{table}

\pagebreak

\section{Discussion of results}
In this section we discuss the results of our simulations in the previous chapter. We give reasons for the performance
of different strategies used by different players.
\subsection{Phase I}
In phase I we used two basic players.
A very basic simple player who used some random numbers and the powerrating of his cards to decide when to raise or to
call. The second player is very similar but uses a more aggressive strategy, which means he'll try to raise more often
and doesn't fold as easily.

Interestingly it is not possible to see a
clear distinction between the two players looking at the results. Since especially for the first round the random
numbers are the only changing number, this could be the reason for optaining very  random results over the course of many
simulations.

\subsection{Phase II}
In phase II we added two new players. One uses the rollout simulation results as his only resource for reasoning about
the bets he is about to make, the other also uses the handstrength-calculations which we implemented. We notice two
things:
\begin{enumerate}
\item The rollout simulation clearly adds a lot of benefit to the players reasoning, even if using it for the late game
phases.
\item The handstrength-calculations seem to provide a very good indication about how good the players hand is, as the
handstrength-player always wins.
\end{enumerate}

Because the phase II players use real game knowledge it seems reasonable to believe that it provides a real advantage
over basic random numbers. The information is especially useful because the player can now fold more often and with a
good reason, thereby he doesn't loose a lot of money because of staying in the game to long.


\subsection{Phase III}
In phase III we added two players making use of an opponent model. The results indicate that the current model, or how
it is used, does not help in winning games of poker against less "evolved" computer players. The phase II player using
the rollout-simulation results and the handstrength calculations is still much better.
We see a number of possible reasons for this:
\begin{enumerate}
    \item We use too broad or too small contexts
    \item The players draw the wrong conclusions from the information gathered in the model
    \item The information gathered by handstrength-calculations is more valuable than our model's data
\end{enumerate}

Interestingly the first model player is much better than model player V2, although they use very similar reasoning
paths. Small changes in the limits of when a player folds/raises/calls seem to have a large effect on the player's
performance compared to the other players. A lot of fine-tuning is necessary to result in a very good poker player.


%\appendix

%\chapter{Patches}\label{appendix}


\chapter{Grammars}\label{appendixGrammars}


%\bibliography{references}

\end{document}
