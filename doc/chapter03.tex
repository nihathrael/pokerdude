\section{Opponent Modelling}
In this section we explain our opponent model and the contexts on which it is based.

\subsection{Model Details}
We have one model per player, which every other player can access. The model contains a mapping for \textit{context} ->
\textit{handstrength} pairs played by the player. The pairs are only recorded if the player is part of the showdown at
the end of a single game. If a context appears multiple times, the average of all recorded handstrengths is used. All
context-handstrength pairs are saved in a \textit{HashMap} for very fast access to the data.

\subsection{Context Details}
We use fairly specific contexts. Our contexts carry the following information to identify them:
\begin{itemize}
    \item The game-state (PreTurn, PreRiver, etc.)
    \item The action the player playen (raise, call, fold)
    \item The player this context is valid for
    \item The current pot-odds, devided into four different bins
    \item The number of raises that have been played in this betting round
    \item The common cards' values on the table, the suite is ignored
\end{itemize}

The pot-odds is discretized into five different bins: pot-odds < 0.1, pot-odds < 0.2, pot-odds < 0.3 and pot-odds >=
0.3. As the pot-odds is a double value, using the exact value would lead to almost infinite numbers of contexts, which
is not very useful for the AI to base decisions upon.

By using these specific contexts we get a great number of different number of contexts. Our hope is to make the
information gathered as useful as possible. 
