\section{Opponent Modelling}
In this section we explain our opponent model and the contexts on which it is based.

\subsection{Model Details}
We have one model per player, which every other player can access. The model contains a mapping for \textit{context} ->
\textit{handstrength} pairs played by the player. The pairs are only recorded if the player is part of the showdown at
the end of a single game. If a context appears multiple times, the average of all recorded handstrengths is used. All
context-handstrength pairs are saved in a \textit{HashMap} for very fast access to the data.

\subsection{Context Details}
We use fairly specific contexts. Our contexts carry the following information to identify them:
\begin{itemize}
    \item The game-state (PreTurn, PreRiver, etc.)
    \item The action the player playen (raise, call, fold)
    \item The player this context is valid for
    \item The common cards' values on the table, the suite is ignored
    \item The current pot-odds, devided into four different bins
    \item The number of raises that have been played in this betting round
\end{itemize}
Having the game-state and the current action in the context seemed natural, as they cleary differentiate the different
situations inside a game and should thus be treated by different contexts. As we expect to have different behaviour per
player, we included the player into the context.

The common cards on the table are one of the main differences between every round, thus it is essential to have this
information present in each context. Because there are many possible combinations of suites for the same values of
cards, we ignore the suites for the common cards. This is a basic form of equivalence class seperation.

The pot-odds is discretized into four different bins: $pot-odds < 0.1$, $pot-odds < 0.2 $, $pot-odds < 0.3$
and $ pot-odds \geq 0.3$. As the pot-odds is a double value, using the exact value would lead to almost infinite numbers of contexts, which
is not very useful for the AI to base decisions upon. As the pot-odds provide a lot of information concerning the risk
the player has when choosing an action, we think that it can be benificial to use it to differentiate between different
contexts.

We also included the number of raises that have been made in a round into the context as we believe that the number of
raises is a good indication of the general game situation, many raises implies that many players believe to have many
cards. This can influence the players decision on which action to take and should thus be included in the context.
