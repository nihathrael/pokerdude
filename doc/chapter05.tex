\section{Discussion of results}
In this section we discuss the results of our simulations in the previous chapter. We give reasons for the performance
of different strategies used by different players.
\subsection{Phase I}
In phase I we used two basic players.
A very basic simple player who used some random numbers and the powerrating of his cards to decide when to raise or to
call. The second player is very similar but uses a more aggressive strategy, which means he'll try to raise more often
and doesn't fold as easily.

Interestingly it is not possible to see a
clear distinction between the two players looking at the results. Since especially for the first round the random
numbers are the only changing number, this could be the reason for optaining very  random results over the course of many
simulations.

\subsection{Phase II}
In phase II we added two new players. One uses the rollout simulation results as his only resource for reasoning about
the bets he is about to make, the other also uses the handstrength-calculations which we implemented. We notice two
things:
\begin{enumerate}
\item The rollout simulation clearly adds a lot of benefit to the players reasoning, even if using it for the late game
phases.
\item The handstrength-calculations seem to provide a very good indication about how good the players hand is, as the
handstrength-player always wins.
\end{enumerate}

Because the phase II players use real game knowledge it seems reasonable to believe that it provides a real advantage
over basic random numbers. The information is especially useful because the player can now fold more often and with a
good reason, thereby he doesn't loose a lot of money because of staying in the game to long.


\subsection{Phase III}
In phase III we added two players making use of an opponent model. The results indicate that the current model, or how
it is used, does not help in winning games of poker against less "evolved" computer players. The phase II player using
the rollout-simulation results and the handstrength calculations is still much better.
We see a number of possible reasons for this:
\begin{enumerate}
    \item We use too broad or too small contexts
    \item The players draw the wrong conclusions from the information gathered in the model
    \item The information gathered by handstrength-calculations is more valuable than our model's data
\end{enumerate}

Interestingly the first model player is much better than model player V2, although they use very similar reasoning
paths. Small changes in the limits of when a player folds/raises/calls seem to have a large effect on the player's
performance compared to the other players. A lot of fine-tuning is necessary to result in a very good poker player.
