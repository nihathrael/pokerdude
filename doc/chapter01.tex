% {{{=================== Introduction ======================


\section{Implementations}

We decided to use an object oriented design with Java to implement the poker simulation and players.

\subsection{Phase I}

\subsubsection{Code structure}

To control the flow of the poker simulation we use one main class \textbf{PokerGame}.
It owns the players and a deck of cards and manages their interactions like giving cards, querying players for actions
and evaluating the chosen actions. Additionaly it possesses information about the status of the game like the common
cards, the current highest bet and credits in the pot.

The Players are stored in a fixed list, additionally only the active players of a round are stored in another list that
is rotated in each round. We thereby implemented that a different player starts each betting round and the big- and
small blind are bet alternatingly. The core procedure in this class is \textit{bettingRound(GameState, Blinds)}. The
parameters specify the phase of the game and whether betting of blinds is required. With this information the active players are
queried to settle their bets and the betting is looped until there are no changes for a whole round. When all betting
rounds are finished, the power ratings for each player are calculated and the credits are awarded to the winner.

In the first phase we implemented a naive class \textbf{Player} that is base-class for all players we implemented later.
It owns the players´ two hand cards and the amount of credits. Further it knows the game object to query certain
information about the game's status for betting decisions. The class provides a function \textit{getBet(GameState, minBet,
force)} that returns the action chosen by the player in the game phase specified by \textit{GameState}. The parameters
\textit{minBet} and \textit{force} indicate basic conditions for the bet the player has to settle on.

To wrap the different actions a player can take in one turn, we introduced an abstract base class \textbf{PokerAction} that
stores the properties of a player action. It contains a procedure that executes the action by calling procedures of
\textbf{PokerGame} like \textit{call(), foldPlayer()} or \textit{raise()}.

The concrete implementations for this class are \textbf{CallAction, RaiseAction} and \textbf{FoldAction}.

The class \textbf{Deck} implements the 52 playing cards available in a poker game. The cards are stored in an
\textit{ArrayList} and also provides the functionality to shuffle the deck. There is a variety of constructors to
support the simulations for the AI Players and a single or more cards can be drawn.

The card itself has no functionality and only stores the suite and the value.

For evaluating sets of cards according to poker rules we had to implement the power ratings newly since we dont use
python. Therefore we introduced 2 classes: \textbf{CardSet} and \textbf{PowerRating}.

A \textbf{PowerRating} stores the rating for a set of cards as described in the lecture and provides functionality to
compare these ratings in order to find out which one is better.

\textbf{CardSet} contains a number of cards and is used to find the best rating for these cards.
This is done by checking if the set contains a constellation of cards (e.g. StraightFlush). If yes the related power
rating is returned and if not the set is checked if it contains the next lower evaluated constellation.

\subsubsection{Betting decisions}

To create simple AI players that profit from some of our preceding methods, we implemented \textbf{PlayerIRaiser} and
\textbf{Player}. In pre flop situations \textbf{PlayerIRaiser} chooses randomly whether to raise or to call since the
only tool of assessment is power ratings and for that five cards are needed. In every situation after pre flop, the
rating for the player cards combined with the available common cards is compared to a "constant" rating. If the cards
are rated less or equal to a pair \textbf{PlayerIRaiser} folds. Otherwise he calls.

\textbf{Player} uses similar betting decisions but with different constants. He folds when the rating is worse than two
pairs and in pre flop situations he raises more often.


\subsection{Phase II}

\subsubsection{Code structure}
The first new feature in phase 2 is pre flop rollout simulations to find a measure for the quality of the two sole hand
cards.
To calculate the pre flop propabilities and to store them in a file we added two more classes: \textbf{RolloutSimulation}
and \textbf{PreFlopPropability}.

As recommended we calculate the pre flop propabilities only for the 169 equivalence
classes. \textbf{PreFlopPropability} stores the information for one prototype of each of the equivalence classes.
Additionaly it is able to encode the information to a string and to parse it from a string. Thus it is easy to read and
write the table containing the propabilities.

The class \textbf{RolloutSimulation} is used to extract the values from the table and has the ability to calculate them
newly. When it is instantiated it checks if a file with the propability information exists. If yes, it is read line by
line and instances of \textbf{PreFlopPropability} are created and stored in a map. If not, a new table is generated 
with the help of many rollout simulations. For each rollout a shuffled \textbf{Deck} without the players cards is
created and the unknown cards are fed from the deck. 

To access the table the procedure \textit{double GetPropabilityFromList(Hand, numberOfPlayers)} is used. The parameter \textit{Hand} provides a list of cards that
specifies the hand for which the pre flop propability is required. Inside the procedure it is mapped onto the prototype
of the fitting equivalence class. The other parameter indicates the number of players and by that the required column of
the table.

The second new feature of phase two is handstrength calculations. For that we added the class \textbf{HandStrength}.
It provides a static procedure \textit{calcHandstrength(playerCards, commonCards, numberOfActivePlayers)}. It being
static makes it independent from the context. Thus it can be used easily by the AI players that profit from the
handstrength calculations, as well as the opponent modeling classes in phase three. The three parameters suffice for the
calculations and can be easily extracted from a \textbf{PokerGame}. After encountering performance issues we added
buffering for the handstrength results using a \textit{HashMap}.

To include the new methods in the poker simulation two new players were equipped with an instance of
\textbf{RolloutSimulation}, which they use to consider the propability in their betting decisions. Only one of them uses
the \textit{calcHandstrength()} methode to be able to assess the profit.


\subsubsection{Betting decisions}

We introduced two new Players that profit from the new features: \textbf{PlayerAI} and \textbf{PlayerAIHandStrength}.
While \textbf{PlayerAI} only uses rollout simulations to calculate the betting behaviour, \textbf{PlayerAIHandStrength}
also employs handstrength calculations. Therefore the betting decisions of \textbf{PlayerAIHandStrength} depend strongly
on the newly revealed common cards during the later rounds.

\textbf{PlayerAI} uses a constant threshold for betting. If the pre flop rating is below 0.3 the player folds. If it is
higher or equal, he bets an amount proportional to the pre flop rating or, if that is below the current bet he calls.
\textbf{PlayerAIHandStrength} considers the ratio between the amount of credits in the pot and the amount needed to stay
in the game, the pot odds, strongly. Only if the hand strength rating is larger than the pot odds the player stays in
the game. If the rating is larger than the pot odds plus a constant amount, the player raises. If it is only larger he
calls. The pre flop betting behaviour is carried over from \textbf{PlayerAI}.

\subsection{Phase III}

\subsubsection{Code structure}

To implement opponent modeling, we introduced one class \textbf{Context} that stores the game situation at a point of
time. The equals method is used in a way that all contexts that differ in a certain range are equal. Thus the way of
implementing the equals method effects on how narrow or wide a context is.
The class \textbf{OpponentModelTable} is instanciated for each player. It records the players actions and if the player
reaches showdown the hand strengths are mapped onto the context bins. For that purpose an additional class
\textbf{ContextAvarage} is used that helps with calculating the average.
As stated before each player is equipped with an \textbf{OpponentModelTable} a context can be created out of a game situation
that can be used to query the map, to fetch the average.
The AI players that use the opponent modeling technique can access each players \textbf{OpponentModelTable} by
extracting the active players from the game and using their \textbf{OpponentModelTable} object.

\subsubsection{Betting decisions}

We introduced two new AI Players in phase three using opponent modeling in their betting decisions. The first one is \textbf{PlayerAIModeling}. For each player it calculates the average estimated hand strength for all actions in the current round. If the players own rating is higher than each of these avarages of estimated hand strengths, in others words the player thinks he should win, the player raises a big amount. If the player does not win against all opponents, but the hand strength is still bigger than the pot odds plus a constant summand, he raises a small amount. 
If the hand stength is only a little higher than the pot odds the player calls and otherwise if it is lower he folds.

\textbf{PlayerAIModeling} calculates the average estimated hand strengths in the same way as the first player, but uses the results in a different way. This player folds if he looses against all opponents if the number of opponents for which results from the opponent model are provided is at least two. Otherwise he behaves in the similar way as \textbf{PlayerAIModeling}. The constant that is added to pot odds to differentiate between raising and calling is bigger and the raised amount is linear to the current bet in the game. 


%}}}

